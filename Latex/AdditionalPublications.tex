\vspace{1cm}
\hrule
\vspace{0.3cm}
\subsection*{\Huge iCVD of a freestanding photoswitchable aero-polymer with an incorporated bridged azobenzene:\\ 3D structure, photoinduced motion, biocompatibility and potential application as photomechanical cell scaffold }
\addcontentsline{toc}{subsection}{\protect \hspace{0.8cm} iCVD of a freestanding photoswitchable aero-polymer with an incorporated bridged azobenzene: 3D structure, photoinduced motion, biocompatibility and potential application as photomechanical cell scaffold}
\hrule
\vspace{1.5cm}

In this work, the previously introduced template of tetrapodal zinc oxide (t-ZnO) was employed for the fabrication of a freestanding photoswitchable aero-polymer by initiated chemical vapour deposition (iCVD). Deposition of 2-hydroxyethyl methacrylate (HEMA) and a photoswitchable bridged azobenzene as cross-linking unit onto the sacrificial t-ZnO template and subsequent template removal, resulted in a highly porous microtubular polymeric network with nanoscopic thin walls, referred to as aero-photoswitch. The photoswitching properties were investigated by illuminating with blue light, revealing a macroscopic motion response. Further studies on the biocompatibility, cell attachment and proliferation demonstrate the potential application as smart photoswitchable scaffold and substrate for cell mechanotransduction.\\\\ 

The results have been summarised in the following manuscript. \\\\

\textbf{Own contribution to the presented article:}
\begin{itemize}
\item Fabrication of t-ZnO templates
\item Fabrication of aero-samples
\item Writing parts of the manuscript
\end{itemize}

\newpage
% \includepdf[pages=-,frame, scale=0.8, pagecommand={}]{paper/aero-photoswitch}
% \includepdf[pages=-,frame, scale=0.8, pagecommand={}]{paper/aero-photoswitch-SI}


\newpage
\vspace{0.3cm}
\hrule
\subsection*{\Huge Sequential treatment with temozolomide plus naturally-derived AT101 as an alternative therapeutic strategy:\\ Insights into chemoresistance mechanisms of surviving glioblastoma cells}
\addcontentsline{toc}{subsection}{\protect \hspace{0.8cm} Sequential treatment with Temozolomide plus naturally-derived AT101 as an alternative therapeutic strategy: Insights into chemoresistance mechanisms of surviving glioblastoma cells}
\hrule
\vspace{1.5cm}

This work presents the results on the sequential treatment of glioblastoma (GBM) cells with temozolomide (TMZ) and AT101, the R-(-)-enantiomer of the naturally occurring cottonseed-derived gossypol. In a complex $in$ $vitro$ co-culture model the chemoresistance mechanisms of primary GBM cells were investigated by analysis of the intracellular effects. The study reveals that GBM cells show chemoresistance mechnisms even when combining chemotherapeutic agents with different effect mechanisms. This requires the development of multimodal treatment strategies involving further agents.\\

The results have been submitted to the Open Access Journal $International$ $Journal$ $of$ $Molecular$ $Sciences$ by MDPI.\\

\textbf{Own contribution to the presented article:}
\begin{itemize}
\item Scientific Discussions
\item Manuscript reviewing and editing
\end{itemize}
% \includepdf[pages=-,frame, scale=0.8, pagecommand={}]{paper/Hellmold et al_IntJMolSci}