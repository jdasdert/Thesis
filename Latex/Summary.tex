In this cumulative thesis three articles presenting three mathematical models responding to questions of different complexity in medical implant development were presented. All three approaches are based on the assumption of continuum theory and were applied in finite element simulations. 

The first article demonstrated the possibility to accurately predict the diffusion controlled release of a drug delivery system consisting of a drug loaded reservoir and a microchannel penetrated membrane. The microchannels of the membrane were approximated by one dimensional interconnections forming a three dimensional network which allowed the simplification of the problem to a one dimensional continuum model using Fickean diffusion. A meshing algorithm creating virtual microchannel networks depending on the fabrication parameters was created and used in three dimensional finite element simulations. The model was verified by comparison of simulation results to experimental \textit{in vitro} measurements, which showed excellent agreement. Finally, it was demonstrated that the model was suitable to support the determination of optimal tuning parameters and can be used to predict the \textit{in vivo} concentration profiles over time, which makes the model a powerful tool for a patient specific customization of the implant.

The second article demonstrated the ability to describe the mechanical behavior for metallic materials exhibiting highly anisotropic stress-strain behavior due to twinning. An efficient numerical algorithm for the treatment of a volume fraction transfer scheme was presented, enabling the simulation of slip, twinning and seconday slip, i.e. slip inside of twins. This approach does not require the resolution of individual twins thus offering the possibility for simulations on a larger scale which is required e.g. in structural investigations of implants. 

While the first two articles each focus on a single aspect of the implant behavior, i.e. diffusion and mechanics, the third article demonstrated the possibility of coupling multiple different interactions. A potential based variational framework was presented, ensuring the conservation of mass, the balances of particles, charge and momentum and consistency with the second law of thermodynamics. The potential structure makes the framework easily adjustable and is suitable to model the degradation of a metallic material. 

In summary the thesis demonstrates the ability 