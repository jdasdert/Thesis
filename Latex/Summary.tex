In this cumulative thesis three articles were presented, showing three mathematical modelling approaches responding to complex questions in medical implant development. All three approaches are based on continuum theory and were applied in finite element simulations. 

The first article demonstrated the possibility to accurately predict the diffusion controlled release of a drug delivery system consisting of a drug loaded reservoir and a microchannel penetrated membrane. The microchannels of the membrane were approximated by one dimensional interconnections forming a three dimensional network which allowed the simplification of the problem to a one dimensional continuum model using Fickean diffusion. A meshing algorithm creating virtual microchannel networks depending on the fabrication parameters was created and used in three dimensional finite element simulations. The model was verified by comparison of simulation results to experimental \textit{in vitro} measurements, which showed excellent agreement. Finally, it was demonstrated that the model was suitable to support the determination of optimal tuning parameters and can be used to predict the \textit{in vivo} concentration profiles over time, which makes the model a powerful tool for a patient specific customization of the implant.

The second article demonstrated an approach to describe the mechanical behavior for metallic materials exhibiting highly anisotropic stress-strain behavior due to twinning, which is a pronounced plastic deformation mechanism especially in metals with hexagonal crystal structure such as magnesium. An efficient numerical algorithm for a volume fraction transfer scheme was presented, enabling the simulation of slip, twinning and secondary slip, i.e. slip inside of twins, without the necessity of resolving individual twins. Thus, this approach allows for simulations on scales larger than the micro-scale, which is required for structural investigations of implants. The model was verified in finite element simulations of single-crystal and polycrystal stress-strain behavior and compared to experimental data, and showed excellent agreement.  

While the first two articles each focused on a single aspect of the implant behavior, i.e. diffusion and mechanics, the third article demonstrated the possibility of coupling multiple different interactions. A electro-chemo-mechanical model was presented, which is based on the variational generalized standard materials framework. The model ensures the conservation of mass, the balances of particles, charge and momentum and consistency with the second law of thermodynamics and proved to be capable to describe the formation and dissolution of a degradation layer. The chosen potential structure makes the framework easily adjustable and can be translated to other research questions involving coupled electro-chemistry coupled to mechanics. 

In summary this thesis demonstrates the ability  