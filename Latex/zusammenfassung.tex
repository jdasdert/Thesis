In der Natur sind hierarchische Netzwerksysteme ein allgegenwärtiges Motiv und für den Transport von Flüssigkeiten, Gasen und Nährstoffen von entscheidender Bedeutung. Diese bioinspirierten Kanalnetzwerke nachzubilden und in weiche Materialien zu integrieren, bietet die Möglichkeit, die Materialeigenschaften anzupassen und ihnen neue Funktionen zu verleihen. Im Rahmen dieser kumulativen Dissertation wurden Templat-basierte Mikro- und Nanostrukturierungsmethoden für Polymere entwickelt, um weiche Durchdringungskomposite mit überlegener Leistungsfähigkeit für verschiedene Anwendungen herzustellen. Die vernetzten Mikrokanalsysteme wurden in die Polymermatrizen durch Inversion der Struktur von tetrapodalen Zinkoxid (t-ZnO) Opfertemplaten eingebracht, die hochporöse, miteinander verbundene Netzwerke bilden.
Nach dem Vorbild natürlich vorkommender mehrschichtiger Kanalwände wurde eine zusätzliche Funktionalisierung der Mikrokanäle durch Beschichtung des t-ZnO mit Graphen erreicht. Dieser Ansatz ist auf andere Nanomaterialien übertragbar und bietet ein Konzept für die Entwicklung multifunktionaler Komposite. Basierend auf diesen Methoden wurden ein neuartiges implantierbares Wirkstofffreisetzungssystem für die lokale Behandlung von Glioblastomen sowie weiche Hydrogel-Aktoren mit bemerkenswerter Leistungsfähigkeit entwickelt, die als zwei Kernthemen in dieser Arbeit behandelt werden. Für das erste Kernthema wurden der Stand der Technik und die aktuelle Forschung zu implantierbaren Wirkstofffreisetzungssystemen für die lokale Behandlung von Glioblastomen in einem Übersichtsartikel zusammengefasst. Auf dieser Grundlage wurde ein Wirkstofffreisetzungssystem, bestehend aus einem Reservoir und einer umgebenden Membran, die von einem Netzwerk von Mikrokanälen durchdrungen ist, entwickelt, das eine diffusionskontrollierte und einstellbare Langzeitfreisetzung von bis zu 60 Tagen für eine patientenspezifische Wirkstoffverabreichung ermöglicht. In einem Proof-of-Concept wurde das Wirkstofffreisetzungssystem weiter modifiziert, um eine steuerbare Freisetzung für eine Wirkstoffabgabe bei Bedarf zu ermöglichen. Im Rahmen des zweiten Kernthemas wurden mit Graphen beschichtete Mikrokanäle in Hydrogele eingebracht, um eine hohe Leitfähigkeit bei sehr geringem Füllstoffgehalt und ein deutlich verbessertes Aktuationsverhalten zu erzielen. Der vorgestellte Ansatz ermöglicht es, die Limitierungen des Wassertransports in thermo-responsiven Hydrogelaktoren zu überwinden und gleichzeitig die weichen und elastischen Eigenschaften sowie die zyklische mechanische Stabilität beizubehalten. Darüber hinaus bieten die elektrisch leitfähigen Hydrogelkomposite auch Möglichkeiten für licht- und elektrisch gesteuerte Aktuationen, die für verschiedene Aktorkonzepte und -designs genutzt wurden. Beide Anwendungen zeigen, dass die Mikro- und Nanostrukturierung von Polymeren vielversprechend ist, um Komposite mit verbesserten Eigenschaften und erweiterter Funktionalität zu entwickeln und den Weg für komplexe multifunktionale weiche und intelligente Materialsysteme zu ebnen, die sich an den Konstruktionsprinzipien der Natur orientieren.