This section provides some additional explanations and interpretations to support the description of the models and equations in the articles. A brief overview of the kinematic description, the fundamental equations ensuring thermodynamic consistency and the material laws used is given below.
\subsection{Kinematic Description of Deformation}\label{sec:kinematics}
The following is a brief summary of the relevant kinematic equations and definitions, which are standard and can be found in many continuum mechanics textbooks\supercite{haupt_continuum_2002,bonet_nonlinear_2008,altenbach_kontinuumsmechanik_2015}. The deformation of a body can be described by the deformation gradient $\FF$.  
\begin{align}
  \FF(\XX,t) = \Grad{\xx} = \frac{\partial \xx}{\partial \XX} 
\end{align}
where $\XX$ is a position vector of a material point in the (undeformed) reference configuration, $\xx(\XX,t)$ is the corresponding position vector in the (deformed) current configuration and $t$ is the time. A graphical interpretation is given in \cref{fig:DeformationGradient}. \\
\begin{figure}[h!]
  \centering
  \includegraphics[]{Figs/DeformationGradient/DeformationGradient.pdf}
  \caption[]{The local deformation of a material can be described by the change in distance and orientation of two material points $P$ and $Q$ in an infinitesimal distance. }
  \label{fig:DeformationGradient}
\end{figure}

With $\FF$ and \mbox{$J=\det{\FF}$} it is possible to map any infinitesimal line element $\rmd\XX$, area $\rmd\AA$, and volume $\rmd V$ from the reference configuration to the current configuration and vice versa:
\begin{align}
  \rmd\xx = \FF\rmd\XX; \qquad \rmd\aa = J \FFinvT \rmd\AA; \qquad \rmd v = J\dV.
\end{align}
Furthermore, the deformation rate is given in terms of the velocity gradient $\ll$ by:
\begin{align}
  \ll = \grad{\dot{\uu}} = \dot\FF\FFinv.
  \label{eq:VelocityGradient}
\end{align}
where \mbox{$\uu(\XX,t)=\xx-\XX$} is the displacement. 
Since deformation can have different origins and mechanisms, it often makes sense to decompose the deformation to be able to describe each deformation mechanism separately. For the deformation gradient the multiplicative decomposition is widely accepted. In scenarios involving chemistry the deformation is often decomposed into an elastic mechanical part and a chemical part, where the chemical part is assumed to be an concentration dependent isotropic swelling, similar to an isotropic thermal expansion.
\begin{align}
  \FF=\FF^\rme\FF^\rmc \qquad \text{with}\qquad \FF^\rmc = (J^\rmc)^{\third}\II.
  \label{eq:FFeFFc}
\end{align} 
Here, $\II$ is the identity tensor and $J^\rmc(C)=\det{\FF^\rmc}$ is the concentration dependent determinant of the chemical deformation. In crystal plasticity on the other hand, the deformation is usually decomposed into an elastic part and an irreversible plastic part, where the plastic part is assumed to be a volume conserving shear deformation:
\begin{align}
  \FF = \FF^\rme\FF^\rmp \qquad \text{with}\qquad \det\FFp = 1
  \label{eq:FFeFFp}
\end{align}
It is often useful to define an intermediate configuration for the decomposition, which in this case can be interpreted as the configuration of virtual elastic unloading, i.e. the deformation that would be obtained if the stress were elastically relieved and a stress-free state were obtained. A graphical interpretation for the two decompositions and the intermediate configuration can be found in Figure~\ref{fig:DecompositionF}. 
\begin{figure}[h!]
  \centering
  \includegraphics[width=\textwidth]{Figs/DecompositionF/DecompositionF.pdf}
  \caption{Common decompositions of the deformation gradient for chemical swelling and crystal plasticity with virtual stress free intermediate configuration.}
  \label{fig:DecompositionF}
\end{figure}

For rate dependent material behavior, the decomposition of the velocity gradient follows from the decomposition of the deformation gradient by inserting Equations~(\ref{eq:FFeFFc})/(\ref{eq:FFeFFp}) into \cref{eq:VelocityGradient}. In case of \cref{eq:FFeFFp} this leads to:
\begin{align}
  \ll = \FFedot\FFeinv + \FFe \FFpdot \FFpinv \FFeinv = \lle + \FFe\LLp\FFeinv.
  \label{eq:DecompositionLL}
\end{align}
This allows the plastic deformation of, for example, a crystal to be described by defining the velocity gradient in the intermediate configuration $\LLp$, with the advantage that elastic distortions and rotations do not need to be taken into account and a given crystal lattice therefore has a well-defined orientation.

\subsection{Balances and Thermodynamic Consistency}
In this section the relevant balance laws are summarized, which are explained in detail elsewhere\supercite{hertel_continuum_2012}. In order to be consistent with the fundamental principles of physics and thermodynamics, balance laws are used, which are always similar in structure. This is examplary shown for an arbitrary quantity $Z$. The rate of change of $Z$ depends on the external supply and internal sources. The corresponding balance equation can be given in integral form or equivalently in the local form as:
\begin{align}
  \dot Z  \ = \ \underbrace{ \intVO \dot z \dV}_{\mathclap{\text{Rate of change}}} \ = \ \underbrace{\intdVO -\JJ^{\rmz} \cdot \NN\dA}_{\text{External supply}} + \underbrace{\intVO \omega^{\rmz} \dV}_{\mathclap{\text{Internal sources}}} 
  \qquad \Leftrightarrow \qquad \dot z = -\Div(\JJ^{\rmz}) + \omega^{\rmz}
  \label{eq:balanceZ}
\end{align} 
where, $z$ is the volume density, $\JJ^\rmz$ is the current density and $\omega^\rmz$ is the internal source density of $Z$ and $\NN$ is the surface normal of the surface increment $\rmd A$. Here and in the following the Lagrangian formulation is used, which corresponds to a description in the reference configuration, i.e. densities are given per unit reference volume or reference area. The transformation from the integral form to the local form uses the divergence theorem and the argument that the balance must hold for an arbitrary volume. A graphical interpretation of \cref{eq:balanceZ} is given in \cref{fig:Balance}.
\begin{figure}[h!]
  \centering
  \includegraphics[width=\textwidth]{Figs/Balance/Balance.pdf}
  \caption[]{Balance of an arbitrary quantity $Z$ can be given for a finite Volume $V$ or locally for an infinitesimal volume $\rmd V$.}
  \label{fig:Balance}
\end{figure}

Depending on the considered effects, the contributions to the balance laws below can be extended or reduced. Here, they are given considering the relevant mechanical, chemical and electric effects considered in this thesis. The first fundamental quantity to be balanced is mass. Defining a system such that it conserves its total mass $m$, the balance equation reads:
\begin{align}
  \dot m = \intVO \dot\rho_0 \dV = 0 \qquad \Leftrightarrow \qquad \dot\rho_0 = 0
  \label{eq:balanceM}
\end{align}
where $\rho_0$ is the mass density per unit reference volume $V_0$. Note that this does not mean that mass can not move nor that the mass density can not change in the current configuration. Any movement of mass is now defined as a displacement $\uu$ (see \cref{fig:DeformationGradient}) which changes the shape of the infinitesimal volume element of the current configuration. The average mass velocity of each material point is thus given by $\dot \uu$. The balances for the number of species $N_i$ considering diffusion and chemical reactions are given by:
\begin{align}
  \begin{aligned}
  \dot N_i  &= \intVO \dot C_i \dV  = \intdVO -\JJ_i \cdot \NN\dA  +  \intVO \sum_r \nu_{ir } R_r \dV \\
            &\Leftrightarrow \qquad \dot C_i = -\Div\JJ_i + \sum_r \nu_{ir } R_r 
  \end{aligned}
  \label{eq:balanceC}
\end{align}  
where $C_i$ is the concentration $\JJ_i$ is the particle current density, $R_r$ is the reaction rate density for a chemical reaction and $\nu_{ir}$ are the stoichiometric coefficients. Similarly, the balance for charge $Q^\rme$, assuming there are no internal sources, is given by:
\begin{align}
  \begin{aligned}
  \dot Q^\rme &= \intVO \dot \rho^\rme_0 \dV = \intdVO \Big(\JJ^\rme-\sum_i Fz_i\,\JJ_i\Big) \cdot \NN\dA \\
            &\Leftrightarrow \qquad \dot \rho^{\rme} = \Div(\JJ^\rme - \sum_i Fz_i\, \JJ_i)
  \end{aligned}
\end{align}
where $\rho^\rme$ is the electric charge density, $\JJ^\rme$ is the electric current due to electrons, $F$ is Faraday's constant and $z_i$ is the charge number of a species. The balance for linear momentum $\pp$, considering that the mass density of the reference configuration does not change (\cref{eq:balanceM}), is given by:
\begin{align}
  \begin{aligned}
    \dot \pp  & = \intVO \rho_0 \,\ddot\xx \dV = \intdVO \hat\tt \dA + \intVO \bb \dV \\
              & \Leftrightarrow  \qquad \rho_0 \ddot\xx = \Div\PP + \bb
    \label{eq:balancep}
  \end{aligned}
\end{align}
where $\hat\tt=\PP\NN$ are tractions on the surface, $\PP$ is the first Piola-Kirchhoff stress tensor and $\bb$ are body forces. The balance for angular momentum $\LL$ reads:
\begin{align}
  \begin{aligned}
    \dot \LL  &= \intVO \xx \times \rho_0 \,\ddot\xx \dV  = \intdVO \xx \times \hat\tt \dA + \intVO \xx \times \bb \dV\\
              &\Rightarrow \qquad \vSS = \vSST
  \end{aligned}
\end{align}
where $\vSS$ is the second Piola-Kirchhoff stress tensor and the linear momentum balance (\cref{eq:balancep}) has been used. The balance for  kinetic energy $E_{\rm kin}$ and internal energy $U$ reads: 
\begin{align}
  \begin{aligned}
  &\begin{aligned}
  \dot E_{\rm kin} + \dot U = \intVO \rho_0(\dot\xx\cdot\ddot\xx + \dot e) \dV &= \intdVO \Big(\hat\tt\cdot\dot\uu -\big(\sum_i e_i \JJ_i+\QQ\big) \cdot \NN \Big) \dA \\
          & \ + \intVO \bb\cdot \dot\uu + \EE\cdot\big(\dot\DD+\JJ^\rme-\sum_i Fz_i\,\JJ_i\big) \dV \\
  \end{aligned} \\
  &\begin{aligned}
    \Leftrightarrow \qquad \rho_0(\dot\xx\cdot\ddot\xx+\dot e) &= \PP:\dot\FF + \bb \cdot \dot\uu - \Div\Big(\sum_i e_i \JJ_i+ \QQ \Big) \\
    & \ + \EE\cdot\Big(\dot \DD+\JJ^\rme-\sum_i Fz_i\JJ_i\Big) 
  \end{aligned}
  \end{aligned}
\end{align}
where $e$ is the specific internal energy, $e_i$ is the molar internal energy of a species, $\QQ$ is an energy flux density, and $\DD$ is the electric displacement. Finally, for the entropy $S$ the imbalance is given by: 
\begin{align}
  \begin{aligned}
    \dot S & = \intVO \rho_0 \dot s \dV \geq \intdVO -\Big(\frac{\QQ}{T} + \sum_i \eta_i\JJ_i\Big) \cdot \NN \dA \\
           & \Leftrightarrow \qquad \dot s \geq -\Div\Big(\frac{\QQ}{T}+\sum_i \eta_i \JJ_i \Big)
  \end{aligned}
\end{align}
where $s$ is the specific entropy, $T$ is the temperature and $\eta_i$ is the molar entropy for a species. In this thesis the processes are assumed to be quasi static ($\ddot\xx=0$) and isothermal ($T=\text{const.}$; $\Grad T = 0$). Further we neglect body forces ($\bb=\zzero$). Thus, the local form of the (im)balance equations used in this thesis therefore can be summarized by:
\begin{align}
  \vphantom{\sum_i}& \dot\rho_0 = 0 \\
  \vphantom{\sum_i}& \dot C_i = -\Div\JJ_i + \sum_r \nu_{ir} R_r \\
  \vphantom{\sum_i}& \dot \rho^{\rme} = \Div(\JJ^\rme - \sum_i Fz_i\, \JJ_i) \\
  \vphantom{\sum_i}& \Div\PP = \zzero \\
  \vphantom{\sum_i}& \vSS = \vSST \\
  \vphantom{\sum_i}& \rho_0\,\dot e = \PP:\dot\FF - \Div\Big(\sum_i e_i \JJ_i+ \QQ \Big)+ \EE\cdot\Big(\dot \DD+\JJ^\rme-\sum_i Fz_i\JJ_i\Big) \\
  \vphantom{\sum_i}& \rho_0\,\dot s \geq -\Div\Big(\frac{\QQ}{T}+\sum_i \eta_i \JJ_i \Big)
\end{align}

Using the definition for free energy densitiy $\psi=\rho_0(e-sT)$, for chemical potential $\mu_i=e_i-\eta_iT$, for the reaction chemical potential difference $(\Delta_R\mu)_r=\sum_i \mu_i\nu_{ir}$ and for the electrochemical potential gradient $\Grad{\mu^\rme_i}=\Grad{\mu_i} + Fz_i\EE$, we can use the above given (im)balance equations to obtain the dissipation equation which we require to be satisfied. 

\begin{align}
  \begin{aligned}
    \mathcal{D} = &\PP:\dot\FF + \sum_i \mu_i\,\dot C_i - \sum_r (\Delta_R\mu)_r \,R_r - \Grad{\mu_i^\rme}\cdot\JJ_i  \\
                + &\EE\cdot\Big(\dot\DD+\JJ^\rme-\sum_i Fz_i\,\JJ_i\Big)- \dot\psi\geq 0
  \end{aligned}
  \label{eq:dissipation}
\end{align}

At this point it is necessary to define functions for the expressions given in the dissipation equation. These functions will define the material behavior and are therefore called material laws. 

\subsection{Material Models}
\subsubsection{Diffusion Controlled Drug Delivery System}
The choice of the material law for the drug delivery system is based on a few assumptions which are shortly explained here. The drug delivery system investigated in this thesis is schematically depicted in \cref{fig:DDS}
\begin{figure}[h!]
  \centering
  \includegraphics{Figs/DDS_schematic/DDS_schematic.pdf}
  \caption{Schematic of the drug delivery system. A drug loaded reservoir is surrounded by a porous membrane interpenetrated by a microchannel network. The channels are assumed to have a constant cross section and are therefore approximated by one dimensional connections.}
  \label{fig:DDS}
\end{figure}

The systems has a drug loaded reservoir which is connected to the outside by a microchannel interpenetrated membrane. It is assumed that the channels can be approximated by a network of tubes, where each tube has a constant cross sectional area. With these assumptions it is possible to describe the microchannels as a three dimensional network of one dimensional interconnections. Further it is assumed that diffusion is the only release mechanism and that the diffusion can be described by Fick's law which is explained in more detail elsewhere\supercite{taylor_multicomponent_1993}. In one dimension it is given by: 
\begin{align}
  j = -D \ \frac{\partial c}{\partial x},
\end{align}
where $c$ is the concentration, $j$ is the current density, $x$ is the position and $D$ is the material dependent diffusion coefficient. Since deformations are not considered in this case, the current configuration is equivalent to the reference configuration. Inserting the material law into the balance equation for the diffusing species (compare \cref{eq:balanceC})
\begin{align}
  \dot c = - \frac{\partial j}{\partial x}
\end{align}
Fick's second law is obtained: 
\begin{align}
  \dot c = D \frac{\partial^2 c}{\partial x^2}.
\end{align}
The solution of this partial differential equation is obtained by the finite element method (FEM) and given in detail in the article in \cref{sec:Article1}. Since it is not discussed in the article, it should be said here that of course Fickean diffusion also satisfies the dissipation equation (compare \cref{eq:dissipation}) which reads for one dimension considering only diffusion of one species.
\begin{align}
  \mathcal{D} = \mu \dot c - \frac{\partial \mu}{\partial x} \, j - \dot \psi \geq 0
\end{align} 
Assuming the total concentration of particles to be approximately constant and the free energy to be a function for a dilute ideal mixture, where only of the concentration of the diffusing species is considered, one has:
\begin{align}
  \partial_c \psi = \mu = \mu^{\circ} + R T \ln{\chi} \qquad \Rightarrow \qquad \frac{\partial \mu}{\partial x} = \frac{R T}{\chi} \frac{\partial \chi}{\partial x} \approx \frac{RT}{c} \frac{\partial c}{\partial x}
\end{align}
where, $\munord$ is the standard Gibb's free energy of formation, $R$ is the gas constant and $\chi$ is the molar fraction we obtain
\begin{align}
  \mathcal{D} = \frac{RTD}{c} \bigg(\frac{\partial c}{\partial x}\bigg)^2 \geq 0
\end{align}
which is always true since $R$, $T$, $D$ and $c$ are always positive.

\subsubsection{Crystal plasticity including twinning}
The equations of the crystal plasticity model follow the standard arguments of crystal plasticity modelling\supercite{roters_crystal_2010}. The plastic deformation rate is given by the velocity gradient in the intermediate configuration $\LLp$ (explained in \cref{sec:kinematics}) and is defined by a shear deformation rate $\nu$, a deformation direction $\dd$ and a plane with normal $\nn$:
\begin{align}
  \LLp = \nu \, \dd\otimes\nn = \nu\MM.
  \label{eq:LLpCrystalPlas}
\end{align}
This description is used for both slip and twinning, which means that although the two mechanisms are profoundly different, mathematically they are treated similar. The direction and plane are combined to the deformation system \mbox{$\MM=\dd\otimes\nn$}. A~schematic interpretation of \cref{eq:LLpCrystalPlas} is depicted in \cref{fig:CrystalPlasticity1}. \\
\begin{figure}[h!]
  \centering
  \includegraphics[width=0.8\textwidth]{Figs/CrystalPlasticity1/CrystalPlasticity1.pdf}
  \caption{Slip and twinning can be describes mathematically similar by a shear, a direction $\dd$ and a plane with normal $\nn$.}
  \label{fig:CrystalPlasticity1}
\end{figure}

However, one key difference is that slip preserves the lattice orientation and can result in arbitrary deformation angles, whereas twinning results in a rotation of the lattice through a well-defined angle. This means that the slip and twinning systems inside a twinned region are also rotated by this angle as shown in \cref{fig:CrystalPlasticity2}. \\
\begin{figure}[h!]
  \centering
  \includegraphics[width=0.55\textwidth]{Figs/CrystalPlasticity2/CrystalPlasticity2.pdf}
  \caption{Twinning leads to a rotation of the lattice in the intermediate configuration, which rotates also the deformation systems.}
  \label{fig:CrystalPlasticity2}
\end{figure}

The overall deformation rate is therefore obtained by a summation over the deformation rates on all existing slip and twinning systems, multiplied by the volume fraction of the respective twin orientation\supercite{kalidindi1998incorporation}:  
\begin{equation}
  \LLp = \underbrace{\sumbO \cb \sumaI \nuab \MMab}_{\text{slip}} + \underbrace{\sumbI \nub \MMb}_{\text{twinning}}.
\end{equation}
Twinning does not only lead to a deformation but also changes the volume fraction~$\cb$ of the respective twin, which is referred to as volume fraction transfer scheme:
\begin{align}
  \cbdot = \frac{\nu_\beta}{\gammatw_\beta}.
\end{align}
Here, $\gammatw_\beta$ defines the constant well-defined shear for a twinning system. The material dependence now comes into play by the slip and twinning systems, and the deformation rates which are assumed to follow a power law (neglecting the indices $\alpha$,$\beta$):
\begin{align}
  \nu = \nu_0 \, \sign{\tau} \left( \frac{\abs{\tau}}{\tau^\rmc} \right)^{p}.
  \label{eq:powerlaw}
\end{align}
The deformation rate depends on a reference rate $\nu_0$, the projection of the stress onto the deformation system $\tau$, a critical stress $\tau^\rmc$ and the exponent~$p$. With the definitions of the right Cauchy-Green tensor of the intermediate configuration $\CCe=\FFeT\FFe$ and Green-Lagrange strain $\EEe=\half(\CCe-\II)$ of the intermediate configuration the stress in the intermediate configuration and its projection are given by:
\begin{align}
  \SSe &= \CCCC:\EEe,  \label{eq:HookesLaw} \\[3pt]
  \tau &= (\CCe\SSe):\MM.
\end{align}
\cref{eq:HookesLaw} is the three dimensional Hooke's law where $\CCCC$ is the fourth order material dependent stiffness tensor. Furthermore, hardening is accounted for by assuming the critical stress to depend on the previous plastic deformation history given by $\gamma_{\rm acc}$.
\begin{align}
  \tauc = \tauc(\gammaacc) \qquad \text{with} \qquad \gammaacc = \int_t \nu \, \rmd t
\end{align}
Finally, he dissipation equation for the purely mechanical case reads:
\begin{align} 
  \mathcal{D} = \PP:\dot\FF - \dot\psi \geq 0
\end{align}
and is in detail discussed in the article in \cref{sec:Article2}. A possible schematic stress-strain diagram for a metal exhibiting easily activated twinning modes and an anisotropic grain orientation is depicted in \cref{fig:MgPlas}.
\begin{figure}[h!]
  \centering
  \includegraphics*[width=0.93\textwidth]{Figs/MgPlas_schematic/MgPlas_schematic.pdf}
  \caption{Schematic stress-strain curve for anisotropic metal with loading direction favorable for easily activated twinning systems. Twinning can only occur until 100 \% has transformed which results in a second elastic regime followed by deformation due to slip. }
  \label{fig:MgPlas}
\end{figure}

For this example we assume a single twinning system and a single slip system. Furthermore, creep is assumed negligible which corresponds to a very large exponent $p$ in the power law. Initially, in the elastic regime Hooke's law describes the linear increase in stress $\SSe$. If the stress projection $\tau$ reaches the critical stress~$\tauc$ of the twinning systems, irreversible plastic deformation starts. In case of twinning as the deformation mechanism, this plastic deformation can only continue until 100~\% of the material has twinned (\mbox{$\cb=1$}). At this point, the stress increases again according to Hooke's law. Note that the slope will in general be different since the crystal lattice is now rotated. Finally, upon reaching the critical stress for slip deformation, again plastic flow occurs. Both deformation mechanisms in general will lead to hardening but may have different impacts.

\subsubsection{Electro-Chemo-Mechanical Degradation}
The processes considered for the degradation of a metallic material in a corrosive medium are schematically depicted in \cref{fig:Corrosion}.
\begin{figure}[h!]
  \centering
  \includegraphics*[width=0.95\textwidth]{Figs/Corrosion/Corrosion.pdf}
  \caption{Schematic representation of processes considered in electro-chemo-mechanical model. Different phases have different properties. Chemical reactions may cause phase transformations, which can cause mechanical stresses. Diffusion of ions leads to a electric fields which can cause electric currents due to electrons.}
  \label{fig:Corrosion}
\end{figure}

In the beginning the metal and medium are in contact and chemical reactions can take place at the interface. A corrosion layer forms which separates the medium and metal from each other. This corrosion layer is assumed to be a new phase with different properties. In general it does not have the same volume which may lead to mechanical stresses. Further it is assumed to be permeable for ions but in general can be non-conducting for electrons, which means that diffusion of ions through the membrane may lead to electric fields forming across the interface. Finally, the degradation of the implant can take place by the corrosion layer growing into the metal while dissolving into the medium. The driving force for the multi-component diffusion are given by the gradient in electro-chemical potential:
\begin{align}
  \jj_i = -\sum_j D_{ij} \, \grad{\mu_j^\rme}
\end{align}
where $D_ij$ are diffusion components consistent with the Maxwell Stefan diffusion. The chemical potential $\mu_i$ is assumed to be given by the law for an ideal mixture: 
\begin{align}
  \mu_i = \munordi + RT \ln{\chi_i}
\end{align}
Chemical reactions are assumed to always have a forward and a backward reaction. 
\begin{align}
  R = k\,\sinh\bigg(\frac{\Delta_R \mu}{\RT}\bigg)
\end{align}
The Cauchy stress $\ssigma=\ssigma^\rmm+\ssigma^\rme$ has a mechanical contribution and an electrical contribution. Since the material is assumed to be isotropic, the mechanical stress $\ssigma^\rmm$ is given by the isotropic Neo-Hookean model: 
\begin{align}
  \ssigma^\rmm = \frac{1}{2J} \Big( \lambda^{\text{Lam\'e}}(\det\CCe-1)\II+2\mu^{\text{Lam\'e}}(\CCe-\II))
\end{align}
where $\lambda^{\text{Lam\'e}}$ and $\mu^{\text{Lam\'e}}$ are the first and second Lam\'e constants.
The electric part is given by the Maxwell-Stress tensor:
\begin{align}
  \ssigma^\rme = \epsilon \Big(\grad\varphi \otimes \grad\varphi + \half (\grad\varphi \cdot \grad\varphi) \II \Big)
\end{align}
The electric displacement field in the current configuration $\tilde \DD$ is assumed to take the commonly known form: 
\begin{align}
  \tilde\DD = -\epsilon \, \grad \varphi
\end{align}
where $\epsilon$ is the material dependent permittivity and $\varphi$ is the electric potential. For electric currents due to electrons $\jj^\rme$ Ohm's law is assumed:
\begin{align}
  \jj^\rme = -\sigma^\rme \, \grad \varphi
\end{align}
where $\sigma^\rme$ is the material specific conductivity.
The dissipation equation is ensured by using the potential based variational framework of generalized standard materials.  
