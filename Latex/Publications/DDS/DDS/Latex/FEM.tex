\section{Finite Element Modeling}
  The approximation with finite elements is obtained starting by reformulating the differential equation in \cref{eq:diffEq} by multiplying with a weight function $v$ and integrating over the volume of the domain:
  \begin{equation}
    \intl v \dcDt A \ \dx = \intl v D \ddcDDx A \ \dx.
  \end{equation}
  Here, $A$ is a constant cross section and $l$ is the total length of the domain.  Note that the weight function $v$ is zero at Dirichlet boundaries but otherwise arbitrary. Using partial integration one obtains:
  \begin{equation}
    \begin{split}
    \intl v A \dcDt \ \dx + \intl \dvDx D A \dcDx \ \dx=  - \left[ v A j \right]_{x=0}^{l}
    \end{split}
    \label{eq:weakForm}
  \end{equation}
  Dividing the domain into finite elements and introducing shape functions $N$ that interpolate between the nodes $i$ the concentration in one element $e$ can be given by:
  \begin{align}
    c(x,t) &= \sumNen \Nie(x) \cie(t) = \NNeT \cce \label{eq:cN}\\
    \dcDx &= \sumNen \dNieDx c_i^e= \BBeT \cce
    \label{eq:dcDxN}
  \end{align}
 For a one dimensional linear element with two nodes the vectors $\NNe$, $\BBe$ and $\cce$ are given by
  \begin{equation}
    \NNe = \begin{pmatrix} N_1^e(x) \\ N_2^e(x) \end{pmatrix}; \qquad 
    \BBe = \begin{pmatrix} \partial N_1^e / \partial x \\ \partial N_2^e / \partial x \end{pmatrix}; \qquad 
    \cce = \begin{pmatrix} c_1^e \\ c_2^e \end{pmatrix}
    \label{eq:NNeBBecce}
  \end{equation}
  The function $v(x)$, its derivative as well as the vectors $\vve$ and $\xxe$ are given in analogy to \cref{eq:cN,eq:dcDxN,eq:NNeBBecce}. Further, it is possible to introduce a reference element in a separate coordinate $\xi$ which is mapped to $x$ by the Jacobian \mbox{$J=\partial x / \partial \xi$}.
  \begin{equation}
    c(\xi,t) = \NNT \cce; \qquad \dcDxi = \BBT \cce; \qquad \BBe = \Jeinv \BB
  \end{equation}
  with
  \begin{equation}
    \NN = \begin{pmatrix} N_1(\xi) \\ N_2(\xi) \end{pmatrix}; \qquad 
    \BB = \begin{pmatrix} \partial N_1 / \partial \xi \\ \partial N_2 /\partial \xi \end{pmatrix}; \qquad 
  \end{equation}
  The Jacobian $\Je$ is obtained by:
  \begin{equation}
    x(\xi) = \NNT\xxe ; \qquad \dxDxi = \BBT \xxe = \Je 
  \end{equation}
  For an element with $-1\le\xi\le 1$ one can define
  \begin{align}
    \CCe &= \intXi A \NN \NNT \Je \dxi&&\approx \sumip A \NN(\xi_i) \NNT(\xi_i) \Je(\xi_i) w_i&& \\
    \KKe &= \intXi D A \BB \BBT \Jeinv \dxi&&\approx \sumip D A \BB(\xi_i) \BBT(\xi_i) \Jeinv(\xi_i) w_i&&
  \end{align}
  Here, the integral is approximated by a summation over integration points with index $i$ and weight $w_i$. The position in the reference element is given by $\xi_i$ and the number of integration points is denoted by $\Nip$. For a one-dimensional element with linear shape functions one integration point at $\xi=0$ and $w=2$ would be sufficient for an exact solution. Now, \cref{eq:weakForm} can be reformulated as follows:
  \begin{equation}
     \sumel \vveT \CCe \dcceDt + \sumel \vveT \KKe \cce = \vvT \FFext \label{eq:FEMe}
  \end{equation}
  The vector $\FFext$ contains all known external forces that are applied at the global nodes. Introducing a matrix $\LLe$ that maps the local nodes of the element to the global nodes (similar to connectivity martrix) one obtains 
  \begin{equation}
    \vve = \LLe \vv; \qquad \cce = \LLe \cc; \qquad \CC = \sumel \LLeT \CCe \LLe; \qquad \KK = \sumel \LLeT \KKe \LLe
  \end{equation}
  and can reformulate \cref{eq:FEMe} as 
  \begin{equation}
    \vvT \CC \dccDt + \vvT \KK \cc = \vvT \FFext \label{eq:FEM}
  \end{equation}
  The discretization in time leads to the following approximation
  \begin{equation}
    \dcceDt \approx \frac{\Delta \cce}{\Dt}  = \frac{\ccenI-\ccen}{\Delta t}
  \end{equation}
  where the index $n$ indicates known quantities from the previous time step and the index $n+1$ indicates quantities of the current time step. The latter is omitted from here on for simplicity.
  Thus, one has: 
  \begin{equation}
    \vvT \left( \frac{\CC}{\Dt}+ \KK \right) \cc = \vvT \left(\FFext + \frac{\CC}{\Dt} \ccn\right).
  \end{equation}

  Note that the entries of $\vv$ are only arbitrary at nodes that have no Dirichlet BC. To eliminate $\vv$ from the equation it is possible to split the vector $\cc$ into a vector $\cc_{\rm u}$, containing the unknowns and zeros at the Dirichlet nodes, and a vector $\cck$, containing the known values at the Dirichlet nodes and zeros everywhere else. Rearranging then yields:
  \begin{equation}
    \vvT \left( \frac{\CC}{\Dt}+\KK\right) \cc_{\rm u} = \vvT \left( \underbrace{\FFext - \KK \cc_{\rm k}}_{\FF} +\frac{\CC}{\Dt}(\ccn-\cc_{\rm k})  \right).
  \end{equation}
  Finally, as $\vv$, $\cc_{\rm u}$ and $\ccn-\cck$ contain zeros at all entries of Dirichlet nodes, it is possible to eliminate all corresponding lines columns. And with $\FF = \FFext - \KK\cck$ one obtains:
  \begin{equation}
    \left(\frac{\CCred}{\Dt}+\KKred\right) \ccred = \FFred + \frac{\CCred}{\Dt} \ccnred.
  \end{equation} 
  
  % \begin{algorithm}
  %   \caption{FEM - 1D time dependent diffusion (simple algorithm)}
  %   \begin{algorithmic}
  %     \State Get element connectivity matrix \texttt{elmt}
  %     \State Define boolean array for Dirichlet nodes \texttt{isDiri}
  %     \State $\KK\gets\zzero$; $\CC\gets\zzero$; $\FF\gets\FFext$
  %     \For{$e=1$, $\Nel$}
  %     \State Get integration point data $\Nip$, $\xi_i$, $w_i$
  %     \State $\CCe = \zzero$; $\KKe=\zzero$; $\ccek=\zzero$
  %     \For{$i=1$, $\Nip$} \Comment Computation of $\CCe$ and $\KKe$
  %       \State Get $\NN(\xi_i)$, $\BB(\xi_i)$ and $\Je(\xi_i)$ at position $\xi_i$
  %       \State $\CCe \gets \CCe+A\NN \NNT \Je w_i$
  %       \State $\KKe \gets \KKe+DA \BB \BBT \Jeinv w_i$
  %     \EndFor
  %     \For{$i=1$, $\Nen$} \Comment Assembly of $\ccek$ and computation of $\FFe$
  %     \If{$\mathtt{isDiri}(\mathtt{elmt}(e,i))$}
  %       \State $\cc^{e}_{{\rm k}, i} = \cc_{{\rm n},\mathtt{elmt}(e,i)}$
  %     \EndIf
  %     \EndFor
  %     \State $\FFe\gets-\KKe\ccek$
  %     \For{$i=1$, $\Nen$} \Comment Assembly of $\KK$, $\CC$ and $\FF$
  %       \For{$j=1$, $\Nen$}
  %         \State $K_{\mathtt{elmt}(e,i),\mathtt{elmt}(e,j)} \gets K_{\mathtt{elmt}(e,i),\mathtt{elmt}(e,j)} + K^e_{ij}$
  %         \State $C_{\mathtt{elmt}(e,i),\mathtt{elmt}(e,j)} \gets C_{\mathtt{elmt}(e,i),\mathtt{elmt}(e,j)} + C^e_{ij}$
  %       \EndFor
  %       \State $F_{\mathtt{elmt}(e,i)} \gets F_{\mathtt{elmt}(e,i)} + F^e_{i}$
  %     \EndFor

  %     \EndFor
  %     \State $\KKred\gets\KK$; $\CCred\gets\CC$; $\FFred\gets\FF$ 
  %     \State Delete lines and columns of Dirichlet nodes
  %     \For{n=1, nsteps}

  %     \EndFor
  %   \end{algorithmic}
  % \end{algorithm}