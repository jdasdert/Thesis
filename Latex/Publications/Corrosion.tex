\ \vspace{1cm} \\
{\Large \bf Article 3}
\normalsize
\vspace{0.3cm}
\hrule
\section*{\Large \centering Framework for an Electro-Chemo-Mechanical Multi-Component Multi-Phase-Field Model}
\addcontentsline{toc}{section}{\protect5 \hspace{0.15cm} Article 3 - Framework for an Electro-Chemo-Mechanical Multi-Component Multi-Phase-Field Model}
\hrule
\vspace{1.5cm}

This Chapter presents the framework for an electro-chemo-mechanical model coupled with a multi-phase-field approach which sets the basis for modelling the formation and dissolution of a corrosion layer in metallic implant materials. The framework ensures conservation of mass, balance of particles, charge and momentum and consistency with the second law of thermodynamics and couples mechanical deformation, chemical swelling, multi-component diffusion, chemical reactions, electric fields, electric currents and phase transformations with a potential based variational approach. The potential approach easily enables the adaption of the model since the individual potentials can be easily expanded, reduced or tailored for the specific problem. The behavior for an exemplary choice of potentials is qualitatively tested with in various 3D finite element simulations.  \\

The manuscript is ready for submission \\

\textbf{Own contribution to the presented article:}
\begin{itemize}
\item Conception of the study
\item Development of the continuum model
\item Implementation of the continuum model
\item Simulations and data analysis
\item Figure creation 
\item Manuscript writing and editing
\end{itemize}