Advances in medicine, materials science, computer science and numerical methods have brought the ideal of a patient-specific implant therapy supported with accurate predictions from digital twins within reach. Upcoming techniques like machine learning and artificial intelligence enable the processing of enormous amounts of data and mathematical models in combination with numerical methods help in creating a better understanding of the highly complex multi-physical processes. This cumulative thesis presents three approaches to describe and simulate the behavior of implant materials using continuum modelling in combination with the finite element method. The first approach models the drug release of a diffusion-controlled drug delivery system, which consists of a drug loaded reservoir surrounded by a porous membrane. The model aims to support the fabrication process by predicting the drug release rate depending on the geometry, porosity and drug loading of the implant. The results from the numerical predictions are compared with experimental measurements and enable the identification of ideal tuning options for a patient-specific therapy. The second approach models the complex mechanical behavior of highly anisotropic metallic implant materials exhibiting twinning as an additional plastic deformation mechanism. A volume fractions transfer scheme allows to include twins without explicitly resolving them, which opens the possibility for simulations on larger scales, which is important for structural analyses of implants. The efficient numerical algorithm significantly reduces the number of unknowns and makes the treatment of slip inside of twins feasible for this approach. The model is verified by comparison to experimental results for single and polycrystal magnesium stress-strain measurements. The third approach is a framework to model the multi-physical processes in corrosion of metallic implant materials. The sophisticated model uses the variational generalized standard materials framework and couples mechanical deformation, chemical swelling, multi-component diffusion, chemical reactions, electric fields, electric currents, multiple phases and phase transformations. Furthermore, the developed framework ensures fundamental laws, such as the balance of mass, charge, momentum and consistency with the second law of thermodynamics. Finite element simulations demonstrate the capability to describe the coupled electro-chemo-mechanical processes involved e.g. in the formation and dissolution of a degradation layer. All three models demonstrate the ability to improve the understanding of implant material behavior and may serve as tools for the optimization of medical implant fabrication processes and therefore pave the way towards full digital twins. 