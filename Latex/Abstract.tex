Advances in medicine, materials science, computer science and numerical methods have brought the ideal of a patient-specific implant therapy supported with accurate predictions from digital twins within reach. Upcoming techniques like machine learning and artificial intelligence enable the processing of enormous amounts of data and mathematical models in combination with numerical methods help in creating a better understanding of the highly complex multi-physical processes. This cumulative thesis presents three approaches to describe the behavior of implant material using continuum modelling in combination with the finite element method. The first approach models the drug release of a diffusion-controlled drug delivery system consisting of a porous membrane and a reservoir. The model aims to support the fabrication process of the drug delivery system by predicting the drug concentration profile over time in the patient, depending on the geometric parameters, porosity and drug loading of the implant. The results from the numerical predictions are compared with experimental measurements and are in good agreement both qualitatively and quantitatively, thus enabling the identification of ideal tuning options for a patient-specific therapy. The second approach models the complex mechanical behavior of a metallic implant material exhibiting highly anisotropic mechanical properties due to twinning as an additional plastic deformation mechanism. The model includes twinning with a volume fraction transfer scheme, which enables to not resolve individual twins, which opens the possibility for simulations on a larger scale. The algorithm for the numerical treatment of secondary plasticity significantly reduces the number of unknowns and allows for an efficient treatment of slip deformation in newly formed twins. The model is verified by comparison to experimental results for single and polycrystal magnesium and shows excellent agreement to experimental stress-strain measurements. The third approach is a framework to model the multi-physical processes in corrosion of metallic implant materials. The model uses the variational generalized standard materials framework and couples mechanical deformation, chemical swelling, multi-component diffusion, chemical reactions, electric fields, electric currents, multiple phases and phase transformations. Finite element simulations demonstrate the capability to represent the coupled electro-chemo-mechanical processes involved in the formation and dissolution of a degradation layer. All three models demonstrate the ability to improve the understanding of implant material behavior and may be seen as tools for the optimization of the medical implant design process which is one fragment on the way to digital twins. 