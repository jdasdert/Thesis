Dank der Fortschritte in Medizin, Materialwissenschaften, Informatik und numerischen Methoden ist das Ideal einer patientenspezifischen Implantattherapie, die durch genaue Vorhersagen von digitalen Zwillingen unterstützt wird, in greifbare Nähe gerückt. Neue Techniken wie maschinelles Lernen und künstliche Intelligenz ermöglichen die Verarbeitung enormer Datenmengen, und mathematische Modelle in Kombination mit numerischen Methoden tragen zu einem besseren Verständnis der hochkomplexen multiphysikalischen Prozesse bei. In dieser kumulativen Dissertation werden drei Ansätze zur Beschreibung des Verhaltens von Implantatmaterialien mit Hilfe der Kontinuumsmodellierung in Kombination mit der Finite-Elemente-Methode vorgestellt. Der erste Ansatz modelliert die Wirkstofffreisetzung eines diffusionsgesteuerten Wirkstofffreisetzungssystems, das aus einem mit Wirkstoff beladenen Reservoir besteht, welches von einer porösen Membran umgeben ist. Das Modell soll den Herstellungsprozess des Implantats unterstützen, indem es die Medikamentenfreisetzungsrate vorhersagt in Abhängigkeit der Geometrie, der Porosität und der Medikamentenbeladung. Die Ergebnisse der numerischen Vorhersagen werden mit experimentellen Messungen verglichen und ermöglichen die Identifizierung von Regulierungsparametern für eine patientenspezifische Therapie. Der zweite Ansatz modelliert das komplexe mechanische Verhalten stark anisotroper metallischer Implantatwerkstoffe, die Zwillinge als zusätzlichen plastischen Verformungsmechanismus aufweisen. Ein Volumenanteilsübertragungsschema ermöglicht es, Zwillinge einzubeziehen, ohne sie explizit aufzulösen, was die Möglichkeit für Simulationen auf größeren Skalen eröffnet, was für Strukturanalysen von Implantaten wichtig ist. Der effiziente numerische Algorithmus reduziert die Anzahl der Unbekannten erheblich und macht die Behandlung von Versetzungsgleiten innerhalb von Zwillingen für diesen Ansatz praktikabel. Das Modell wird durch den Vergleich mit experimentellen Ergebnissen für ein- und polykristalline Magnesium Spannungs-Dehnungs Messungen verifiziert. Der dritte Ansatz ist ein Rahmenwerk zur Modellierung der multiphysikalischen Prozesse bei der Korrosion von metallischen Implantatwerkstoffen. Das Modell nutzt den Rahmen der generalisierten Standardmaterialien und verbindet mechanische Verformung, chemische Volumenänderungen, Mehrkomponentendiffusion, chemische Reaktionen, elektrische Felder, elektrische Ströme, mehrere Phasen und Phasenumwandlungen. Darüber hinaus gewährleistet der entwickelte Rahmen grundlegende Gesetze wie das Gleichgewicht von Masse, Ladung und Impuls sowie die Übereinstimmung mit dem zweiten Hauptsatz der Thermodynamik. Finite-Elemente-Simulationen zeigen die Fähigkeit, die gekoppelten elektro-chemo-mechanischen Prozesse zu beschreiben, die z. B. bei der Bildung und Auflösung einer Degradationsschicht ablaufen. Alle drei Modelle zeigen, dass sie zu einem besseren Verständnis des Verhaltens von Implantatmaterialien beitragen und als Werkzeuge für die Optimierung medizinischer Implantatfertigungsprozesse dienen können, und ebnen daher den Weg zur Entwicklung vollständiger digitalen Zwillingen. 