Dank der Fortschritte in Medizin, Materialwissenschaften, Informatik und angewandter Numerik ist das Ideal einer patientenspezifischen Implantattherapie, die durch genaue Vorhersagen von digitalen Zwillingen unterstützt wird, in greifbare Nähe gerückt. Neue Techniken wie maschinelles Lernen und künstliche Intelligenz ermöglichen die Verarbeitung enormer Datenmengen, und mathematische Modelle in Kombination mit numerischen Methoden tragen zu einem besseren Verständnis der hochkomplexen multiphysikalischen Prozesse bei. In dieser Arbeit werden drei Ansätze vorgestellt, die das Materialverhalten von Implantaten mit Hilfe von Kontinuumsmodellen in Kombination mit der Finite-Elemente-Methode beschreiben. In dieser kumulativen Dissertation werden drei Ansätze zur Beschreibung des Verhaltens von Implantatmaterial mit Hilfe der Kontinuumsmodellierung in Kombination mit der Finite-Elemente-Methode vorgestellt. Der erste Ansatz modelliert die Wirkstofffreisetzung eines diffusionsgesteuerten Wirkstoffabgabesystems, das aus einer porösen Membran und einem Reservoir besteht. Das Modell soll den Herstellungsprozess des Medikamentenabgabesystems unterstützen, indem es das Profil der Medikamentenkonzentration im Patienten über die Zeit vorhersagt, abhängig von den geometrischen Parametern, der Porosität und der Medikamentenbelastung des Implantats. Die Ergebnisse der numerischen Vorhersagen werden mit experimentellen Messungen verglichen und stimmen sowohl qualitativ als auch quantitativ gut überein, was die Identifizierung idealer Tuning-Optionen für eine patientenspezifische Therapie ermöglicht. 