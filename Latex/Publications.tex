\vspace{1cm}
\Large {Chapter 3.1}
\normalsize
\vspace{0.3cm}
\hrule
\subsection*{\Large \centering Fabrication and Modelling of a Reservoir-Based Drug Delivery
System for Customizable Release}
\addcontentsline{toc}{subsection}{\protect3.1 \hspace{0.15cm} Fabrication and Modelling of a Reservoir-Based Drug Delivery
System for Customizable Release}
\vspace{0.3cm}
\hrule
\vspace{1.5cm}

This Chapter presents the fabrication, characterization and modelling of a drug delivery system (DDS) with diffusion controlled release. The DDS consists of a a drug loaded reservoir surrounded by a microchannel penetrated porous polydimethylsiloxane (PDMS) membrane. The fabrication method is based on a sacrificial ZnO-network template and makes the DDS customizable in reservoir geometry, membrane geometry and porosity and drug loading. The diffusion-controlled release through the microchannel membrane is experimentally measured and mathematically modelled for different fabrication parameters. The 1D continuum model is applied in 3D finite element simulations and shows good agreement to the experimental results. The model is used to predict the concentration profiles over time for examplary \textit{in vivo} conditions demonstrating the customizability of the system for a patient-specific therapy. Finally, a proof of concept is presented for a triggerable release by functionalisation of the microchannels with a thermo-responsive hydrogel. \\

The results have been published in the journal \textit{Pharmaceutics}.\\

\textbf{Own contribution to the presented article:}
\begin{itemize}
\item Conception of the study
\item Development and implementation of the continuum model, the finite element solver and the meshing algorithm for an artificial microchannel membrane
\item Development and implementation of the experimental data analysis software 
\item Implant parameter studies by simulation  
\item Figure creation 
\item Manuscript writing and editing
\end{itemize}

\newpage
% \includepdf[pages=-,frame, scale=0.8, pagecommand={}]{Figs/Publication_DDS/Publication_DDS}

%%%%%%%%%%%%%%%%%%%%%%%%%%%%%%%%%%%%%%%%%%%%%%%%%%%%%%%%%%%%%%%%%%%%%%%%%%%%%%%%%%%%%%%%%%%%%%%%%%%%%%%%%%%%%%%%%%%%%%%%%%%%%%%%%%%%

\newpage
\Large {Chapter 3.2}
\normalsize
\vspace{0.3cm}
\hrule
\subsection*{\Large \centering Efficient numerical strategies for an implicit
volume fraction transfer scheme for single \\ crystal plasticity including twinning and secondary plasticity on the example of magnesium}
\addcontentsline{toc}{subsection}{\protect3.2 \hspace{0.15cm} Efficient numerical strategies for an implicit volume fraction transfer scheme for single crystal plasticity including twinning and secondary plasticity on the example of magnesium}
\hrule
\vspace{1.5cm}

This Chapter presents a model for single crystal plasticity mechanical deformation including plasticity by slip and twinning and secondary plasticity i.e. slip in newly formed twins. The model makes use of the multiplicative decomposition of the deformation gradient, uses a rate dependent power law for the plastic deformation and a volume fraction transfer scheme for the formation of twins. For the numerical treatment of the large amount of unknowns in the volume fraction transfer scheme, an efficient algorithm is introduced. The advantage of this approach is that twins don't need to be explicitly resolved which allows modelling on larger scales, which is important for investigating the structural behavior of medical implants. The result of the model is a highly anisotropic stress strain behavior, which is observable im metallic implant materials such as magnesium due to the hexagonal structure and the easily activated twinning modes. The model is tested in single crystal and polycrystal tension and compression simulations and compared to experimental results for magnesium and shows excellent agreement. \\

The results have been published in the journal \textit{Numerical Methods in Engineering}. \\

\textbf{Own contribution to the presented article:}
\begin{itemize}
\item Conception of the study
\item Implementation and adjustments of the continuum model
\item Simulations and data analysis
\item Figure creation 
\item Manuscript writing and editing
\end{itemize}


%%%%%%%%%%%%%%%%%%%%%%%%%%%%%%%%%%%%%%%%%%%%%%%%%%%%%%%%%%%%%%%%%%%%%%%%%%%%%%%%%%%%%%%%%%%%%%%%%%%%%%%%%%%%%%%%%%%%%%%%%%%%%%%%%%%%


\newpage
\Large {Chapter 3.3}
\normalsize
\vspace{0.3cm}
\hrule
\subsection*{\Large \centering Framework for an electro-chemo-mechanical multi-component multi-phase-field model}
\addcontentsline{toc}{subsection}{\protect3.3 \hspace{0.15cm} Framework for an electro-chemo-mechanical multi-component multi-phase-field model}
\hrule
\vspace{1.5cm}

This Chapter presents the framework for an electro-chemo-mechanical model coupled with a multi-phase-field approach which sets the basis for modelling the formation and dissolution of a corrosion layer in metallic implant materials. The framework ensures conservation of mass, balance of particles, charge and momentum and consistency with the second law of thermodynamics and couples mechanical deformation, chemical swelling, multi-component diffusion, chemical reactions, electric fields, electric currents and phase transformations with a potential based variational approach. The potential approach easily enables the adaption of the model since the individual potentials can be easily expanded, reduced or tailored for the specific problem. The behavior for an exemplary choice of potentials is qualitatively tested with in various 3D finite element simulations.  \\

The manuscript is ready for submission \\

\textbf{Own contribution to the presented article:}
\begin{itemize}
\item Conception of the study
\item Development of the continuum model
\item Implementation of the continuum model
\item Simulations and data analysis
\item Figure creation 
\item Manuscript writing and editing
\end{itemize}


\newpage
% \includepdf[pages=-,frame, scale=0.8, pagecommand={}]{Figs/Publication_MgPlas/Publication_MgPlas}
% \includepdf[pages=-,frame, scale=0.8, pagecommand={}]{paper/pharmaceutics-1650660-supplementary}








