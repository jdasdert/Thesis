\subsection{Biomaterials and Medical Implants}

Biomaterials have been used by humans for at least 4000 years. Cases of replaced teeth, replaced bone parts and sutured fractures were found throughout history for example in ancient Egypt, India, Greece and South African tribes\supercite{hernigou_history_2017,al-shalawi_biodegradable_2023,balamurugan_corrosion_2008}. Although there are several definitions of the term biomaterial, which have changed over time \supercite{marin_biomaterials_2020}, one definition that found consensus among many experts in 2018 reads\supercite{ghasemi-mobarakeh_key_2019}:  \\ 
\begin{center}
  \begin{minipage}{0.9\textwidth}
    Biomaterial: 
    \textit{
      A material designed to take a form that can direct, through interactions with living systems, the course of any therapeutic or diagnostic procedure\\
    }
  \end{minipage}
\end{center}
The widespread use of biomaterials in medical therapy began at the transition from the 19th to the 20th century when the development of antiseptic and aseptic techniques drastically reduced the risk of infection of surgical intervention \supercite{hernigou_history_2017,balamurugan_corrosion_2008,gilbert_medical_2012}. The initial use of metallic alloys mainly in dentistry, was soon followed by internal fixation of fractured bone with metallic plates and screws\supercite{hernigou_history2_2017} and later by replacement of joints and bone parts\supercite{ratner_introduction_1997,fernandez_de_grado_bone_2018}. New surgical procedures and imaging techniques improved the success of these therapies and accelerated research and development of new biomaterials and implants. Nowadays, biomaterials include all sorts of metals, polymers, ceramics and composite materials\supercite{park_biomaterials_2007}. Heart Valve Prostheses, total hip replacement prostheses, intraocular lenses, ventricular assist devices and drug delivery systems are just a few examples of medical procedures that are carried out regularly and with great success\supercite{ratner_introduction_1997,bharadwaj_overview_2021}. However, understanding and controlling the complex corrosion mechanisms in the body has been a major challenge from the beginning. Corrosion can compromise the mechanical integrity and functionality of an implant and may result in the potentially toxic release of corrosion products. Although the advantages of highly corrosive materials such as Magnesium alloys, which are completely degradable by the human metabolism, were recognised early on, controlling the degradation process proved to be problematic \supercite{witte_history_2010}. As a result, research has long focused on developing materials that are highly inert, reducing the complexity and number of variables that need to be considered for successful therapy. For decades biocompatibility basically meant the property of a material to have as little influence on the surrounding organism as possible\supercite{gilbert_medical_2012}. But until today scientists were not successful in developing an implant material which is completely inert to corrosion in the body over the period of a lifetime, which means that implants need to be removed or replaced after some time. This always poses risks that come with additional surgeries and is especially complicated in delicate areas such as the brain. Furthermore, there are many cases in which an implant is needed only temporarily. Thus in recent years research has rediscovered the idea of biodegradable implants and redefined again the understanding of biocompatibility\supercite{ghasemi-mobarakeh_key_2019}: \\
\begin{center}
  \begin{minipage}{0.9\textwidth}
  Biocompatibilty: 
  \textit{
    The ability of a material to perform with an appropriate host response in a specific application.\\
  }
\end{minipage}
\end{center}
Although the challenges of the highly complex mechanisms of corrosion are still remaining, the advancements in medicine, materials science, computer science and numerical methods nowadays provides a much larger variety of tools to understand and predict the biological and multi-physical processes involved. Since ... numerical methods such as the finite element method allow for complicated . Recently with advances in machine learning algorithms and artificial intelligence it is possible to process enormous amounts of data and predict systems of highest complexity with increasing accuracy. Thus it is not surprising that researchers have started to work on digital twins, which are supposed to be able to precisely predict the implant behavior in the body \supercite{cellina_digital_2023} and is a pathway to patient specific therapy.  

Biotribocorrosion
FEM modelling review patient specific modelling \supercite{pankaj_patientspecific_2013}



\subsection{Multi-Physical Continuum Modelling}

Continuum modelling  
The core assumption of continuum theory is the assumption that it is possible to adequately describe a material as a continuous medium without considering discrete quantities such as atoms or molecules. 
 Instead discrete quantities are averaged and can be given in terms of densities (e.g. concentrations). 

\subsection{Aim of this thesis}
This thesis was written in the context of the Research Training Group "Materials for Brain"  
In this cumulative thesis three different approaches for modelling medical implants and implant materials are presented. 


% 
%Until \hl{time} implant materials were chosen to have a large resistivity to the highly corrosive environment of the human body. Potentially harmful interactions with the biological system were therefore often minimized, which allowed to serve a medical purpose over a long period. Examples are given by \hl{...}. However, there is almost no control over the implant once inside the body. Although the interactions are kept at a minimum with increasing time the chance for complications occurring dramatically increases. This often requires to remove or replace the implant after some time. While this is always problematic since surgery always poses a risk it is close to impossible in highly sensitive areas such as the brain. To circumvent the problem of multiple surgeries in recent time research on implants that undergo controlled degradation has increased, shifting the focus from minimizing harmful interactions to designing beneficial interactions. Examples are \hl{...}. Due to the high complexity of the problem, requires a high understanding of the multi-physical and chemical processes involved. Furthermore, 

% \subsection{Continuum Modelling}


% This thesis shows 

% Medical implants: from inert to biodegradable, from general to patient specific
