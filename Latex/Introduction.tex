\subsection{Biomaterials and Medical Implants}

The use of biomaterials date back to antiquity. Cases of replaced teeth, replaced bone parts and sutured fractures were found throughout history for example in ancient Egypt, India, Greece and South African tribes \supercite{hernigou_history_2017,al-shalawi_biodegradable_2023,balamurugan_corrosion_2008}. Although there are several definitions of the term biomaterial, which have changed over time \supercite{marin_biomaterials_2020}, one definition that found consensus among many experts in 2018 reads \supercite{ghasemi-mobarakeh_key_2019}:  \\ 
\begin{center}
  \begin{minipage}{0.9\textwidth}
    Biomaterial: 
    \textit{
      A material designed to take a form that can direct, through interactions with living systems, the course of any therapeutic or diagnostic procedure\\
    }
  \end{minipage}
\end{center}
The widespread use of biomaterials in medical therapy began at the transition from the 19th to the 20th century when the development of antiseptic and aseptic techniques drastically reduced the risk of infection of surgical intervention \supercite{hernigou_history_2017,balamurugan_corrosion_2008,gilbert_medical_2012}. The initial use of metallic alloys mainly in dentistry, was soon followed by internal fixation of fractured bone with metallic plates and screws \supercite{hernigou_history2_2017} and later by replacement of joints and bone parts \supercite{ratner_introduction_1997,fernandez_de_grado_bone_2018}. New surgical procedures and imaging techniques improved the success of these therapies and accelerated research and development of new biomaterials and implants. Nowadays, biomaterials include all sorts of metals, polymers, ceramics and composite materials \supercite{park_biomaterials_2007}. Heart Valve Prostheses, total hip replacement prostheses, intraocular lenses, ventricular assist devices and drug delivery systems are just a few examples of medical procedures that are carried out regularly and with great success \supercite{ratner_introduction_1997,bharadwaj_overview_2021}. However, understanding and controlling the complex corrosion mechanisms in the body has been a major challenge from the beginning. Corrosion can compromise the mechanical integrity and functionality of an implant and may result in the potentially toxic release of corrosion products \supercite{kamachimudali_corrosion_2003,ali_biocompatibility_2020}. Although the advantages of highly corrosive materials such as Magnesium alloys, which are completely degradable by the human metabolism, were recognised early on, controlling the degradation process proved to be problematic \supercite{witte_history_2010}. As a result, research has long focused on developing materials that are highly inert, reducing the complexity and number of variables that need to be considered for successful therapy. For decades biocompatibility basically meant the property of a material to have as little influence on the surrounding organism as possible \supercite{gilbert_medical_2012,jacobs_corrosion_1998}. But scientists are still working to extend the lifetime of implants, because in most cases corrosion cannot be completely prevented, which means that implants have to be removed or replaced after some time \supercite{ali_biocompatibility_2020,tranquillo_surface_2020}. This always poses a risk that comes with surgery and is especially problematic in delicate areas such as the brain. Furthermore, there are many cases in which an implant is needed only temporarily. Thus in recent years research has rediscovered the idea of biodegradable implants and redefined again the understanding of biocompatibility \supercite{ghasemi-mobarakeh_key_2019}: \\
\begin{center}
  \begin{minipage}{0.9\textwidth}
  Biocompatibilty: 
  \textit{
    The ability of a material to perform with an appropriate host response in a specific application.\\
  }
\end{minipage}
\end{center}
Although the challenges of the highly complex mechanisms of corrosion are still present, the advances in medicine, materials science, computer science and numerical methods nowadays provides a much larger variety of tools to understand and predict the biological and multi-physical processes involved \supercite{wang_biodegradable_2020,prakasam_biodegradable_2017}. Mathematical modelling in combination with numerical methods such as the finite element method allow for complex calculations, while the increasing computational power makes it possible to include more and more degrees of freedom \supercite{menicucci_toothimplant_2002,kladovasilakis_finite_2020,alemayehu_three-dimensional_2021,salaha_biomechanical_2023}. Recently, machine learning algorithms and artificial intelligence additionally make it possible to process enormous amounts of data and thus have the potential to predict systems of highest complexity \supercite{revilla-leon_artificial_2023,suwardi_machine_2022}. Therefore it is not surprising that researchers believe in the possibility to create full digital twins in the near future, which are capable to precisely describe the implant behavior in the body \supercite{cellina_digital_2023,pankaj_patientspecific_2013,erol_digital_2020}, which would be a milestone for a superior and patient specific therapy.  

\subsection{Aim of this thesis}
This cumulative thesis was written in the context of the Research Training Group "Materials for Brain", aiming for the localized treatment of brain diseases with medical implants. The aim of this thesis is to provide mathematical models that describe implants and implant materials, which ultimately can be used as parts of digital twins that can accurately predict the implant behavior in the body. The three presented articles show three modelling approaches using continuum modelling in combination with the finite element method. The first article investigates the diffusion controlled drug release of a drug delivery system with the aim to link the fabrication parameters to the obtained drug concentration profiles in the patient. The second article investigates the mechanical behavior for metallic materials with the aim to provide a tool for a complex structural analysis of metallic implants exhibiting twinning. The third article aims to demonstrate how to couple different influences to a complex multi-physical model, which is able to describe the highly nonlinear degradation process in metallic implant materials. 
