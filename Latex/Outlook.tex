\subsection{Diffusion Model of the Drug Delivery System}
The time dependent release rate of the drug delivery model in the presented article was tested \textit{in vitro}. The simulated results of the model being in good agreement with the experimental data shows the capability to model the release without any further influences. However, the prediction for the concentration profile \textit{in vivo} at the moment does not consider the possibility that the release behavior will be different if the implant is in contact with living tissue. The direct contact on the surface as well as growth of cells into the membrane are possible factors that might delay the release. Furthermore, the meshing algorithm that creates the artificial network structure by one dimensional interconnections meshing algorithm of the network structure shows promising results for 

\subsection{Crystal Plasticity Model}
The crystal plasticity model was verified with experimental results for pure Magnesium. Tests with Magnesium alloys and a coupling to 

\subsection{Electro-Chemo-Mechanical Multi-Component Multi-\\Phase Model}
Electro-chemo-mechanical framework was tested qualitatively for reasonable behavior. The next steps would involve to test and compare the model to experimental results of a degrading metallic implant where a degradation layer is forming and dissolving over time. Furthermore, the model is not limited to degradation applications but could also be investigated in other electro-chemo-mechanical applications such as batteries. 