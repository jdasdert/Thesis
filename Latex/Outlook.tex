\subsection*{Diffusion Model of the Drug Delivery System}
It was demonstrated that that the diffusion model is able to predict the release profiles \textit{in vitro} of the customizable drug delivery system correctly. However, up to this point the simulation results are not compared to measurements \textit{in vivo}, which means that the assumption that the model can predict the drug concentration in the patient has not yet been proven. The simulations for the concentration profiles \textit{in vivo} at the moment do not consider the possibility that the release behavior will change if the implant is in contact with living tissue. The direct contact on the surface as well as growth of cells into the membrane are possible factors that could delay the release, which would make an adaption of the diffusion model necessary. Furthermore, since the fabrication method of the network structure has been used in other applications\supercite{schutt_electrically_2021,reimers_multifunctional_2023}, the algorithm creating the artificial network structures could be further tested and validated using other models predicting e.g. electrical conductivity. 

\subsection*{Crystal Plasticity Model}
At this point, the crystal plasticity model has been verified mainly with experimental results for pure magnesium and some estimates for magnesium-silver alloys. However, since the model is not limited to any specific metal, the comparison with other metals and alloys remains a future prospect. Additionally, especially applications with an anisotropic crystal structure such as sputter-deposited thin-films could be used to further test the predictive power of the preseted model. Finally, the possibility to couple the model to other physical and electrochemical processes can be investigated. For a coupling with the electro-chemo-mechanical framework presented in this work, the crystal plasticity model would have to be formulated with a potential, which could then be included, adding a plastic deformation mechanism.  

\subsection*{Electro-Chemo-Mechanical Multi-Component Multi-\\Phase Model}
For the electro-chemo-mechanical framework reasonable behavior and consistency with well known laws was proven qualitatively. Thus, the next steps would involve testing and comparing the model to experimental results of a degrading metallic implant where a degradation layer is forming and dissolving over time. Furthermore, the model is not limited to degradation applications but could also be investigated in other electro-chemo-mechanical applications. One example is battery modelling where a similar but simpler approach for modelling the mechanical stresses of lithium-silicon anodes during lithiation and delithiation has already been used in another work by the author\supercite{dittmann_framework_2023}. 