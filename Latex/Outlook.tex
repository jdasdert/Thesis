\subsection*{Diffusion Model of the Drug Delivery System}
It was demonstrated that that the drug delivery system is highly customizable and the diffusion model is able to predict the release profile \textit{in vitro} correctly. However up to now the simulation results are not compared to measurements \textit{in vivo} which means that the assumption that the model can predict the drug concentration in the patient is not yet proven. The prediction for the concentration profile \textit{in vivo} at the moment does not consider the possibility that the release behavior will be different if the implant is in contact with living tissue. The direct contact on the surface as well as growth of cells into the membrane are possible factors that could delay the release, which would make an adaption of the diffusion model necessary. Furthermore, the algorithm that creates the artificial network structure of one dimensional interconnections could be further tested and improved by using the created geometries with models of heat flow or electric conduction in cases where the sample was created with the same sacrificial template approach. 

\subsection*{Crystal Plasticity Model}
The crystal plasticity model was verified with experimental results for pure Magnesium. Tests with Magnesium alloys and a coupling to 

\subsection*{Electro-Chemo-Mechanical Multi-Component Multi-\\Phase Model}
Electro-chemo-mechanical framework was tested qualitatively for reasonable behavior. The next steps would involve to test and compare the model to experimental results of a degrading metallic implant where a degradation layer is forming and dissolving over time. Furthermore, the model is not limited to degradation applications but could also be investigated in other electro-chemo-mechanical applications such as batteries. 